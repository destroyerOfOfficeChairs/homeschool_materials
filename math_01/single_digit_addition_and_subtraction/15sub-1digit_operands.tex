\documentclass{article}

\usepackage[margin=2cm]{geometry}
\usepackage{tabularray}

\newcommand\numinit{\edef\num{\fpeval{randint(1,9)}}}
\newcommand\binaryinit{\edef\binary{\fpeval{randint(0,1)}}}
\newcommand\operator{
  \binaryinit
  \ifnum\binary>0 + \else - \fi
  }

\newcommand\sortxy{
  \ifnum\y>\x
    \let\temp\y
    \let\y\x
    \let\x\temp
  \fi
  }

\newcommand\addsubone{
  \numinit
  \let\x\num
  \numinit
  \let\y\num
% \binaryinit
%  \ifnum\binary<1 \sortxy \fi
  \sortxy
  \textbf{
  \begin{tabular}{c@{\,}c}
    & \x \\
%    \ifnum\binary>0 + \else - \fi & \y \\
    - & \y \\
    \hline
  \end{tabular}
  }
}

\pagestyle{empty}

\begin{document}

\LARGE

\begin{centering}

15 SUBTRACTION PROBLEMS~~~~~~~~~~~
Date:

\SetTblrInner{rowsep=1cm}

\begin{tblr}{width=0.8\textwidth,colspec={XXXXX}}
  
  \addsubone & \addsubone & \addsubone & \addsubone & \addsubone \\ 
  \addsubone & \addsubone & \addsubone & \addsubone & \addsubone \\ 
  \addsubone & \addsubone & \addsubone & \addsubone & \addsubone 
  
\end{tblr}

\end{centering}

\end{document}
